\section{Analiza}

Przykładowe analizy kostki OLAP zostały wykonane przy użyciu narzędzia Cube Browser dostępnego w SQL Server Management Studio oraz w tabelach przestawnych w programie Microsoft Excel. Poniżej przedstawiono wyniki analiz odpowiadających na kluczowe pytania biznesowe dotyczące wpływu różnych czynników na wyniki studentów i obciążenia studentów i nauczycieli.

\begin{figure}[H]
  \centering
    \includegraphics[width=.8\linewidth]{img/browser.png}
    \caption{Przykładowa analiza w narzędziu Cube Browser}
    \label{fig:browser}
\end{figure}

\noindent
Na rysunku \ref{fig:browser} przedstawiono interfejs narzędzia Cube Browser, w którym przeprowadzono przykładową analizę obciążenia studenta, zdawalości oraz średniej ocen. Analiza uwzględnia wymiary: semestr, forma zaliczenia oraz specjalność studenta. 

\clearpage
\subsection{Pytania analityczne}

\subsubsection{Wpływ typu kursu na średnią ocen i zdawalność}

Pierwsza analiza dotyczy zależności między typem kursu a osiąganymi wynikami studentów. Na rysunku \ref{fig:analysis-1} przedstawiono średnią ocen oraz wskaźnik zdawalności (pass rate) w zależności od kategorii kursu.

\begin{figure}[H]
  \centering
    \includegraphics[width=.8\linewidth]{img/analysis_1.png}
    \caption{Średnia ocen i zdawalność w zależności od typu kursu}
    \label{fig:analysis-1}
\end{figure}

\subsubsection{Wpływ tytułu naukowego prowadzącego na wyniki studentów}

Druga analiza bada zależność między tytułem naukowym prowadzącego a średnią ocen oraz zdawalnością. Wykorzystano hierarchię drill-down w wymiarze tytułów: od skróconego tytułu (title) do pełnej nazwy (title long). Wyniki przedstawiono na rysunku \ref{fig:analysis-2}.

\begin{figure}[H]
  \centering
    \includegraphics[width=.5\linewidth]{img/analysis-2.png}
    \caption{Średnia ocen i zdawalność w zależności od tytułu prowadzącego (hierarchia drill-down)}
    \label{fig:analysis-2}
\end{figure}

\subsubsection{Wpływ jednostki prowadzącego na wyniki studentów}

Kolejna analiza sprawdza, czy jednostka organizacyjna prowadzącego (wydział, instytut) ma wpływ na wyniki studentów. Zastosowano hierarchię drill-down: wydział (faculty) → instytut (institute). Wyniki zaprezentowano na rysunku \ref{fig:analysis-3}.

\begin{figure}[H]
  \centering
    \includegraphics[width=.5\linewidth]{img/analysis-3.png}
    \caption{Średnia ocen i zdawalność w zależności od jednostki prowadzącego (hierarchia drill-down)}
    \label{fig:analysis-3}
\end{figure}

\subsubsection{Zależność wyników od formy zaliczenia i roku studiów}

Analiza przedstawiona na rysunku \ref{fig:analysis-4} bada zdawalność, średnią ocen oraz liczbę ocen w zależności od typu formy zaliczenia (egzamin lub kolokwium) oraz roku studiów studenta.

\begin{figure}[H]
  \centering
    \includegraphics[width=.5\linewidth]{img/analysis-4.png}
    \caption{Zdawalność, średnia ocen i liczba ocen w zależności od formy zaliczenia i roku studiów}
    \label{fig:analysis-4}
\end{figure}

\subsubsection{Wpływ obciążenia nauczyciela na wyniki studentów}

Najważniejsza analiza dotyczy wpływu obciążenia dydaktycznego nauczyciela na wyniki studentów. Badano średnią ocen, zdawalność, liczbę studentów oraz liczbę egzaminów w zależności od semestru (zimowy W, letni S), poziomu obciążenia oraz tytułu prowadzącego. Wyniki zaprezentowano na rysunku \ref{fig:analysis-6}.

\begin{figure}[H]
  \centering
    \includegraphics[width=.7\linewidth]{img/analysis-6.png}
    \caption{Wpływ obciążenia nauczyciela na wyniki studentów w zależności od semestru i tytułu}
    \label{fig:analysis-6}
\end{figure}
