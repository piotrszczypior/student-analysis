\section{SSAS}

Po zakończeniu procesu ETL, dane zostały wykorzystane do budowy kostki OLAP przy pomocy narzędzia SQL Server Analysis Services (SSAS). Celem tego etapu było umożliwienie efektywnej analizy ocen w wielu przekrojach analitycznych, takich jak cechy studentów, prowadzących oraz kursów.

\subsection{Źródło danych i widok}

Po zdefiniowaniu połączenia został stworzony widok (Data Source View), który zawiera tabele faktów oraz tabele wymiarów. Widok ten stanowi podstawę dla kostki OLAP i definiuje relacje pomiędzy tabelami.

\begin{figure}[H]
    \centering
    \includegraphics[width=.8\linewidth]{img/schema.png}
    \caption{Schemat gwiazdy po transformacjach ETL - tabela faktów Fact\_grades oraz tabele wymiarów}
    \label{fig:star-schema}
\end{figure}

\noindent
Schemat gwiazdy (rys. \ref{fig:star-schema}) składa się z centralnej tabeli faktów \texttt{Fact\_grades} oraz dodatkowych tabel dotyczących obciążenia: \texttt{teacher\_semester\_workload} i \texttt{student\_semester\_workload}, wraz z otaczającymi je tabelami wymiarów.

\subsection{Struktura kostki OLAP}
Następnie została stworzona kostka OLAP, a przy pomocy kreatora zostały zdefiniowane wymiary oraz miary analityczne.

\subsubsection{Wymiary}
Wymiary zostały zdefiniowane poprzez kreator wymiarów \texttt{Dimension Wizard}. Każdy wymiar reprezentuje określony aspekt analizy i umożliwia grupowanie oraz filtrowanie danych według różnych kryteriów.

\begin{figure}[H]
  \centering
    \includegraphics[width=.4\linewidth]{img/cube-dim.png}
    \caption{Zdefiniowane wymiary w kostce OLAP}
    \label{fig:cube-dims}
\end{figure}

\clearpage
\noindent
Poniżej znajduje się opis zdefiniowanych wymiarów (rys. \ref{fig:cube-dims}):

\begin{itemize}
  \item Wymiar \texttt{Dim Semester} - wymiar zbudowany na podstawie tabeli \texttt{dim\_semester}
  \item Wymiar \texttt{Dim Students} - wymiar zbudowany na podstawie tabeli \texttt{dim\_students}
  \item Wymiar \texttt{Dim Courses} - wymiar zbudowany na podstawie tabeli \texttt{dim\_courses}
  \item Wymiar \texttt{Dim Teachers} - wymiar zbudowany na podstawie tabeli \texttt{dim\_teachers}
  \item Wymiar \texttt{Dim Teacher Titles} - wymiar zbudowany na podstawie tabeli \texttt{dim\_teacher\_titles}
  \item Wymiar \texttt{Dim Exam Type} - wymiar zbudowany na podstawie kolumny \texttt{exam} z tabeli \texttt{fact\_grades}
  \item Wymiar \texttt{Dim Teacher Semester Workload Tier} - wymiar zbudowany na podstawie kolumny \texttt{workload\_tier} z tabeli \texttt{teacher\_semester\_workload}
\end{itemize}

\subsubsection{Hierarchie w wymiarach}

W wybranych wymiarach zostały zdefiniowane hierarchie, które umożliwiają wielopoziomową nawigację po danych. Hierarchie pozwalają na analizę danych na różnych poziomach szczegółowości. 

\noindent
Przykładem takiej hierarchii jest hierarchia zdefiniowana w wymiarze \texttt{Dim Semester}. Hierarchia ta umożliwia analizę danych w następujących poziomach przedstawionych na rysunku \ref{fig:hierarchy}. ilustruje ona działanie operacji drill-down, gdzie użytkownik może przejść od danych zagregowanych rocznych do bardziej szczegółowych semestralnych danych.

\begin{figure}[H]
    \centering
      \begin{subfigure}{0.45\textwidth}
        \centering
        \includegraphics[width=0.4\linewidth]{img/time-hierarchy.png}
        \caption{Hierarchia czasu w wymiarze Semester}
        \label{fig:time-hierarchy}
    \end{subfigure}
    \begin{subfigure}{0.45\textwidth}
        \centering
        \includegraphics[width=0.6\linewidth]{img/hierarchy.png}
        \caption{Przykładowe dane w hierarchii z operacją drill-down}
        \label{fig:hierarchy}
    \end{subfigure}
    \caption{Hierarchia czasu w kostce OLAP}
    \label{fig:cube-hierarchies}
\end{figure}



\subsubsection{Grupy miar}

Następnie wymiary zostały dodane do nowo utworzonej kostki OLAP. W zakładce \texttt{Cube Structure} zostały zdefiniowane tabele faktów jako nowe grupy miar (\texttt{Measure Group}). Każda grupa miar reprezentuje zestaw powiązanych ze sobą metryk analitycznych pochodzących z określonej tabeli faktów.

\begin{figure}[H]
  \centering
    \includegraphics[width=.5\linewidth]{img/cube-measures.png}
    \caption{Grupy miar zdefiniowane w kostce OLAP}
    \label{fig:cube-measures}
\end{figure}

Na rysunku \ref{fig:cube-measures} przedstawiono zdefiniowane grupy miar, które zawierają podstawowe metryki takie jak liczba ocen, suma punktów oraz liczba pozytywnych wyników.

\subsubsection{Miary wyliczane}

Oprócz podstawowych miar pochodzących bezpośrednio z tabel faktów, za pomocą funkcji \texttt{Calculated Member} w zakładce \texttt{Calculations} zostały dodane miary wyliczane w języku MDX (Multidimensional Expressions). Miary wyliczane zdefiniowane są na podstawie istniejących miar.
\clearpage
\vspace{0.4cm}
\noindent
\textbf{Average Grade}

\noindent
Miara wyliczająca średnią arytmetyczną ocen poprzez podzielenie sumy ocen przez liczbę ocen:

\begin{minted}{sql}
  [Measures].[Grade] / [Measures].[Fact Grades Count]
\end{minted}

\vspace{0.4cm}
\noindent
\textbf{Pass Rate}

\noindent
Miara obliczająca procent pozytywnych ocen w stosunku do wszystkich przyznanych ocen:

\begin{minted}{sql}
  [Measures].[Pass] / [Measures].[Fact Grades Count]
\end{minted}


